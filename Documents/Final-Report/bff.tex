\section{Boundary First Flattening}
In this section, I will get into details of my project on boundary first flattening. The algorithm is based on differential geometry. This section is organized as follows: I will first introduce the theoretical background. Then I will give details on my implementation.

\subsection{Theoretical Background}

\subsubsection{Conformal Maps}
In complex analysis,  conformal maps are equivlent to holomorphic maps, that is, the differentials of these map satisfy Cauchy-Riemann equation:
\begin{equation}
\frac{\partial f}{\partial \bar{z}} = 0
\end{equation}

One important property is that all comformal maps are harmonic maps, that is, 
\begin{equation}
\Delta f = 0
\end{equation}
But the reverse direction is not true.

\subsubsection{Poisson Problems}
Poisson problem is stated as follows:

Dirichlet-type condition:
\begin{equation}
\begin{split}
\Delta a &= \phi \ \ \ & on\  M\\
a &= g , & on\ \partial M
\end{split}
\end{equation}
Or Riemann-type condition:
\begin{equation}
\begin{split}
\Delta a &= \phi \ \ \ & on\  M\\
\frac{\partial a}{\partial n} &= h , & on\ \partial M 
\end{split}
\end{equation}

We will use discrete Poisson equation on a mesn $M$:

\begin{equation}
\left(\begin{matrix}
A_{II} & A_{IB}\\
A_{BI} & A_{BB}
\end{matrix}\right) a
= \left(\begin{matrix}
\phi_I\\
\phi_B - h\end{matrix}\right)
\label{eq:relation}
\end{equation}

where $M$ is the cotangent Laplacian matrix.

Known $A$, Dirichlet-type conditions and Riemann-type conditions can be converted to each other: Given $h$, we simply solve the Poisson equation and set 
\begin{equation}
g = \Lambda^*_\phi h = a_B,
\end{equation}
Given $g$, we can convert it to $h$  by
\begin{equation}
h = \Lambda_\phi g = \phi_B - A_{IB}^TA_{II}^{-1}(\phi_I - A_{IB}g) - A_{BB}g
\end{equation}


\subsubsection{Cherrier Formula}
Cherrier formula \cite{CHERRIER1984154} is the core fact which induces the BFF algorithm. 

For a conformal map $f: M \rightarrow \tilde{M}$, we have 
\begin{equation}
\begin{split}
\Delta u = K - e^{2u} \tilde{K} \ \ \ &on\ &M\,\\
\frac{\partial u}{\partial n} = k - e^{u}\tilde{k} \ \ \ &on\     &\partial M
\end{split}
\end{equation}
where $u$ is conformal factor, $K$,$\tilde{K}$ are Gauss curvature of the source surface and  the target surface, $k, \tilde{k}$ is geodesic curvature of the source surface and the target surface.


The discrete version is the root of the algorithm:
\begin{equation}
\begin{split}
Au &=   \Omega - \tilde\Omega\  &on\  Int\  M\\
h &= k - \title{k}\ \  &on\ \partial M 
\end{split}
\label{eq:poisson}
\end{equation}
where $\Omega_i = 2\pi - \sum_{ijk \in F}\theta_i^{jk}$ defined on interior vertices, $k_i = \pi - \sum_{ijk \in F}\theta_i^{jk}$ defined on boundary vertices, which is the exterior angle at $v_i$. Note that $\Omega$ is defined to be zero at boundary vertices.



\section{Algorithm}

The pipeline of the BFF algorithm is as follows:

1) If boundray conformal factor ${u}$ is known, we compute bounadry target $k$ by: $$ \tilde{k} = k -\Lambda_{\Omega} u.$$
If target boundary $\tilde{k}$ is known, we compute boundary conformal factor $u$ by:: $$u = \Lambda_{\Omega}^{*}(k - \tilde{k}).$$

2) Define new edge lengths as:
$$l_{ij}^* = e^{\frac{u_i + u_j}{2}}l_{ij}.$$ With $\tilde{k}$, which is exterior angles, we can integrate these two data into a closed loop.

3) Extend the loop into interior conformally.


\subsection{Loop Integration}
Usually, directly integrate $\tilde{k}$ over $l_{ij}^*$ won't give a closed loop, we should scale edge length









